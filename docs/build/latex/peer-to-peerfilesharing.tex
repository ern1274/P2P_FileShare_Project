%% Generated by Sphinx.
\def\sphinxdocclass{report}
\documentclass[letterpaper,10pt,oneside,english,openany]{sphinxmanual}
\ifdefined\pdfpxdimen
   \let\sphinxpxdimen\pdfpxdimen\else\newdimen\sphinxpxdimen
\fi \sphinxpxdimen=.75bp\relax
\ifdefined\pdfimageresolution
    \pdfimageresolution= \numexpr \dimexpr1in\relax/\sphinxpxdimen\relax
\fi
%% let collapsible pdf bookmarks panel have high depth per default
\PassOptionsToPackage{bookmarksdepth=5}{hyperref}

\PassOptionsToPackage{booktabs}{sphinx}
\PassOptionsToPackage{colorrows}{sphinx}

\PassOptionsToPackage{warn}{textcomp}
\usepackage[utf8]{inputenc}
\ifdefined\DeclareUnicodeCharacter
% support both utf8 and utf8x syntaxes
  \ifdefined\DeclareUnicodeCharacterAsOptional
    \def\sphinxDUC#1{\DeclareUnicodeCharacter{"#1}}
  \else
    \let\sphinxDUC\DeclareUnicodeCharacter
  \fi
  \sphinxDUC{00A0}{\nobreakspace}
  \sphinxDUC{2500}{\sphinxunichar{2500}}
  \sphinxDUC{2502}{\sphinxunichar{2502}}
  \sphinxDUC{2514}{\sphinxunichar{2514}}
  \sphinxDUC{251C}{\sphinxunichar{251C}}
  \sphinxDUC{2572}{\textbackslash}
\fi
\usepackage{cmap}
\usepackage[T1]{fontenc}
\usepackage{amsmath,amssymb,amstext}
\usepackage[english]{babel}



\usepackage{tgtermes}
\usepackage{tgheros}
\renewcommand{\ttdefault}{txtt}



\usepackage[Bjarne]{fncychap}
\usepackage{sphinx}

\fvset{fontsize=auto}
\usepackage{geometry}


% Include hyperref last.
\usepackage{hyperref}
% Fix anchor placement for figures with captions.
\usepackage{hypcap}% it must be loaded after hyperref.
% Set up styles of URL: it should be placed after hyperref.
\urlstyle{same}

\addto\captionsenglish{\renewcommand{\contentsname}{Modules:}}

\usepackage{sphinxmessages}
\setcounter{tocdepth}{1}


        \usepackage{titlesec}
        \titlespacing*{\chapter}{0pt}{-30pt}{20pt}
        \titlespacing*{\section}{0pt}{-10pt}{10pt}
        \usepackage{etoolbox}
        \patchcmd{\chapter}{\thispagestyle{plain}}{}{}{}
    

\title{Peer\sphinxhyphen{}to\sphinxhyphen{}Peer File Sharing}
\date{Apr 14, 2025}
\release{1.0}
\author{Joshua Talbot and Ethan Nunez}
\newcommand{\sphinxlogo}{\vbox{}}
\renewcommand{\releasename}{Release}
\makeindex
\begin{document}

\ifdefined\shorthandoff
  \ifnum\catcode`\=\string=\active\shorthandoff{=}\fi
  \ifnum\catcode`\"=\active\shorthandoff{"}\fi
\fi

\pagestyle{empty}
\sphinxmaketitle
\pagestyle{plain}
\sphinxtableofcontents
\pagestyle{normal}
\phantomsection\label{\detokenize{index::doc}}


\sphinxAtStartPar
Welcome to the documentation for the Python\sphinxhyphen{}based peer\sphinxhyphen{}to\sphinxhyphen{}peer file sharing system.

\sphinxAtStartPar
This project mimics basic BitTorrent\sphinxhyphen{}style behavior using a custom protocol.


\chapter{Contents}
\label{\detokenize{index:contents}}
\sphinxstepscope


\section{Modules}
\label{\detokenize{modules:modules}}\label{\detokenize{modules::doc}}

\subsection{p2p\_command}
\label{\detokenize{modules:module-p2p_command}}\label{\detokenize{modules:p2p-command}}\index{module@\spxentry{module}!p2p\_command@\spxentry{p2p\_command}}\index{p2p\_command@\spxentry{p2p\_command}!module@\spxentry{module}}\index{exchange\_data() (in module p2p\_command)@\spxentry{exchange\_data()}\spxextra{in module p2p\_command}}

\begin{fulllineitems}
\phantomsection\label{\detokenize{modules:p2p_command.exchange_data}}
\pysigstartsignatures
\pysiglinewithargsret
{\sphinxcode{\sphinxupquote{p2p\_command.}}\sphinxbfcode{\sphinxupquote{exchange\_data}}}
{\sphinxparam{\DUrole{n}{peers}}\sphinxparamcomma \sphinxparam{\DUrole{n}{peer\_name}}\sphinxparamcomma \sphinxparam{\DUrole{n}{file\_id}}\sphinxparamcomma \sphinxparam{\DUrole{n}{receiver}}\sphinxparamcomma \sphinxparam{\DUrole{n}{address}}}
{}
\pysigstopsignatures
\sphinxAtStartPar
Sends an EXCH\_REQ to the remote peer.
The remote peer’s receiver will queue a file\sphinxhyphen{}sending job in ‘exch\_req\_queue’.

\end{fulllineitems}

\index{get\_index\_path() (in module p2p\_command)@\spxentry{get\_index\_path()}\spxextra{in module p2p\_command}}

\begin{fulllineitems}
\phantomsection\label{\detokenize{modules:p2p_command.get_index_path}}
\pysigstartsignatures
\pysiglinewithargsret
{\sphinxcode{\sphinxupquote{p2p\_command.}}\sphinxbfcode{\sphinxupquote{get\_index\_path}}}
{\sphinxparam{\DUrole{n}{exch\_id}}}
{}
\pysigstopsignatures
\sphinxAtStartPar
Given a file ID, return the absolute file path.
\begin{quote}\begin{description}
\sphinxlineitem{Parameters}\begin{itemize}
\item {} 
\sphinxAtStartPar
\sphinxstyleliteralstrong{\sphinxupquote{exch\_peer}} \textendash{} the peer requesting the file (not used currently)

\item {} 
\sphinxAtStartPar
\sphinxstyleliteralstrong{\sphinxupquote{exch\_id}} \textendash{} the file ID being requested

\end{itemize}

\sphinxlineitem{Returns}
\sphinxAtStartPar
the file path as a string, or empty string if not found

\end{description}\end{quote}

\end{fulllineitems}

\index{p2p\_command\_line() (in module p2p\_command)@\spxentry{p2p\_command\_line()}\spxextra{in module p2p\_command}}

\begin{fulllineitems}
\phantomsection\label{\detokenize{modules:p2p_command.p2p_command_line}}
\pysigstartsignatures
\pysiglinewithargsret
{\sphinxcode{\sphinxupquote{p2p\_command.}}\sphinxbfcode{\sphinxupquote{p2p\_command\_line}}}
{\sphinxparam{\DUrole{n}{name}}\sphinxparamcomma \sphinxparam{\DUrole{n}{port}}}
{}
\pysigstopsignatures
\sphinxAtStartPar
Main interface for the P2P system.
Handles user input and executes commands.

\end{fulllineitems}

\index{peer\_discovery() (in module p2p\_command)@\spxentry{peer\_discovery()}\spxextra{in module p2p\_command}}

\begin{fulllineitems}
\phantomsection\label{\detokenize{modules:p2p_command.peer_discovery}}
\pysigstartsignatures
\pysiglinewithargsret
{\sphinxcode{\sphinxupquote{p2p\_command.}}\sphinxbfcode{\sphinxupquote{peer\_discovery}}}
{\sphinxparam{\DUrole{n}{my\_port}}\sphinxparamcomma \sphinxparam{\DUrole{n}{my\_name}}}
{}
\pysigstopsignatures
\sphinxAtStartPar
Uses the tracker to discover other peers in the network.
Returns a dictionary mapping ‘peer\_id’ to (ip, port).

\end{fulllineitems}

\index{print\_index() (in module p2p\_command)@\spxentry{print\_index()}\spxextra{in module p2p\_command}}

\begin{fulllineitems}
\phantomsection\label{\detokenize{modules:p2p_command.print_index}}
\pysigstartsignatures
\pysiglinewithargsret
{\sphinxcode{\sphinxupquote{p2p\_command.}}\sphinxbfcode{\sphinxupquote{print\_index}}}
{\sphinxparam{\DUrole{n}{peer\_addr}\DUrole{o}{=}\DUrole{default_value}{None}}\sphinxparamcomma \sphinxparam{\DUrole{n}{soc}\DUrole{o}{=}\DUrole{default_value}{None}}}
{}
\pysigstopsignatures
\sphinxAtStartPar
Displays the list of available files from this peer.
Uses a separate socket to avoid interference with receiver.

\end{fulllineitems}

\index{print\_menu() (in module p2p\_command)@\spxentry{print\_menu()}\spxextra{in module p2p\_command}}

\begin{fulllineitems}
\phantomsection\label{\detokenize{modules:p2p_command.print_menu}}
\pysigstartsignatures
\pysiglinewithargsret
{\sphinxcode{\sphinxupquote{p2p\_command.}}\sphinxbfcode{\sphinxupquote{print\_menu}}}
{}
{}
\pysigstopsignatures
\sphinxAtStartPar
Prints the command menu for the user.

\end{fulllineitems}

\index{process\_exchange\_requests() (in module p2p\_command)@\spxentry{process\_exchange\_requests()}\spxextra{in module p2p\_command}}

\begin{fulllineitems}
\phantomsection\label{\detokenize{modules:p2p_command.process_exchange_requests}}
\pysigstartsignatures
\pysiglinewithargsret
{\sphinxcode{\sphinxupquote{p2p\_command.}}\sphinxbfcode{\sphinxupquote{process\_exchange\_requests}}}
{\sphinxparam{\DUrole{n}{exch\_req\_queue}}\sphinxparamcomma \sphinxparam{\DUrole{n}{address}}\sphinxparamcomma \sphinxparam{\DUrole{n}{soc}}}
{}
\pysigstopsignatures
\sphinxAtStartPar
Continuously monitors ‘exch\_req\_queue’ for (file\_id, peer\_address) tuples
and starts a Sender to serve that file to ‘peer\_address’.

\end{fulllineitems}

\index{register\_with\_tracker() (in module p2p\_command)@\spxentry{register\_with\_tracker()}\spxextra{in module p2p\_command}}

\begin{fulllineitems}
\phantomsection\label{\detokenize{modules:p2p_command.register_with_tracker}}
\pysigstartsignatures
\pysiglinewithargsret
{\sphinxcode{\sphinxupquote{p2p\_command.}}\sphinxbfcode{\sphinxupquote{register\_with\_tracker}}}
{\sphinxparam{\DUrole{n}{tracker\_host}}\sphinxparamcomma \sphinxparam{\DUrole{n}{tracker\_port}}\sphinxparamcomma \sphinxparam{\DUrole{n}{peer\_host}}\sphinxparamcomma \sphinxparam{\DUrole{n}{peer\_port}}\sphinxparamcomma \sphinxparam{\DUrole{n}{peer\_name}}}
{}
\pysigstopsignatures
\sphinxAtStartPar
Connects to the tracker and registers the current peer.
Returns a list of other peers in the network.

\end{fulllineitems}

\index{start\_tracker() (in module p2p\_command)@\spxentry{start\_tracker()}\spxextra{in module p2p\_command}}

\begin{fulllineitems}
\phantomsection\label{\detokenize{modules:p2p_command.start_tracker}}
\pysigstartsignatures
\pysiglinewithargsret
{\sphinxcode{\sphinxupquote{p2p\_command.}}\sphinxbfcode{\sphinxupquote{start\_tracker}}}
{\sphinxparam{\DUrole{n}{host}\DUrole{o}{=}\DUrole{default_value}{\textquotesingle{}0.0.0.0\textquotesingle{}}}\sphinxparamcomma \sphinxparam{\DUrole{n}{port}\DUrole{o}{=}\DUrole{default_value}{9000}}}
{}
\pysigstopsignatures
\sphinxAtStartPar
Starts the tracker server that listens for incoming peer registrations.
It adds peers to a global set and returns the list of known peers (excluding the caller).

\end{fulllineitems}



\subsection{receiver\_rdt}
\label{\detokenize{modules:module-receiver_rdt}}\label{\detokenize{modules:receiver-rdt}}\index{module@\spxentry{module}!receiver\_rdt@\spxentry{receiver\_rdt}}\index{receiver\_rdt@\spxentry{receiver\_rdt}!module@\spxentry{module}}\index{Receiver (class in receiver\_rdt)@\spxentry{Receiver}\spxextra{class in receiver\_rdt}}

\begin{fulllineitems}
\phantomsection\label{\detokenize{modules:receiver_rdt.Receiver}}
\pysigstartsignatures
\pysiglinewithargsret
{\sphinxbfcode{\sphinxupquote{class\DUrole{w}{ }}}\sphinxcode{\sphinxupquote{receiver\_rdt.}}\sphinxbfcode{\sphinxupquote{Receiver}}}
{\sphinxparam{\DUrole{n}{soc}}\sphinxparamcomma \sphinxparam{\DUrole{n}{peer\_files}\DUrole{o}{=}\DUrole{default_value}{None}}}
{}
\pysigstopsignatures
\sphinxAtStartPar
Bases: \sphinxcode{\sphinxupquote{object}}

\sphinxAtStartPar
Receiver class that can handle multiple files simultaneously.

\sphinxAtStartPar
For each inbound file, we store its ‘base\_seq’, ‘max\_seq’, and ‘packets’
inside self.active\_files{[}file\_id{]}, for example:
\begin{quote}
\begin{description}
\sphinxlineitem{self.active\_files{[}file\_id{]} = \{}
\sphinxAtStartPar
‘base\_seq’: \sphinxhyphen{}1,
‘max\_seq’:  \sphinxhyphen{}1,
‘packets’:  {[}{]}

\end{description}

\sphinxAtStartPar
\}
\end{quote}

\sphinxAtStartPar
When a new chunk arrives for ‘file\_id’, we place it at index {[}seq \sphinxhyphen{} base\_seq{]}.
Once we see seq == \sphinxhyphen{}1, we know the sender is done sending that file, and we
call finalize\_file(file\_id) to write out the chunks and remove the entry.
\index{packets (receiver\_rdt.Receiver attribute)@\spxentry{packets}\spxextra{receiver\_rdt.Receiver attribute}}

\begin{fulllineitems}
\phantomsection\label{\detokenize{modules:receiver_rdt.Receiver.packets}}
\pysigstartsignatures
\pysigline
{\sphinxbfcode{\sphinxupquote{packets}}}
\pysigstopsignatures
\sphinxAtStartPar
Array of received decoded data

\end{fulllineitems}

\index{soc (receiver\_rdt.Receiver attribute)@\spxentry{soc}\spxextra{receiver\_rdt.Receiver attribute}}

\begin{fulllineitems}
\phantomsection\label{\detokenize{modules:receiver_rdt.Receiver.soc}}
\pysigstartsignatures
\pysigline
{\sphinxbfcode{\sphinxupquote{soc}}}
\pysigstopsignatures
\sphinxAtStartPar
socket that receiver uses to bind and receive data over

\end{fulllineitems}

\index{ip (receiver\_rdt.Receiver attribute)@\spxentry{ip}\spxextra{receiver\_rdt.Receiver attribute}}

\begin{fulllineitems}
\phantomsection\label{\detokenize{modules:receiver_rdt.Receiver.ip}}
\pysigstartsignatures
\pysigline
{\sphinxbfcode{\sphinxupquote{ip}}}
\pysigstopsignatures
\sphinxAtStartPar
ip address to receive data from

\end{fulllineitems}

\index{port (receiver\_rdt.Receiver attribute)@\spxentry{port}\spxextra{receiver\_rdt.Receiver attribute}}

\begin{fulllineitems}
\phantomsection\label{\detokenize{modules:receiver_rdt.Receiver.port}}
\pysigstartsignatures
\pysigline
{\sphinxbfcode{\sphinxupquote{port}}}
\pysigstopsignatures
\sphinxAtStartPar
port number to receive data from

\end{fulllineitems}

\index{base\_seq (receiver\_rdt.Receiver attribute)@\spxentry{base\_seq}\spxextra{receiver\_rdt.Receiver attribute}}

\begin{fulllineitems}
\phantomsection\label{\detokenize{modules:receiver_rdt.Receiver.base_seq}}
\pysigstartsignatures
\pysigline
{\sphinxbfcode{\sphinxupquote{base\_seq}}}
\pysigstopsignatures
\sphinxAtStartPar
the lowest sequence number to index by

\end{fulllineitems}

\index{max\_seq (receiver\_rdt.Receiver attribute)@\spxentry{max\_seq}\spxextra{receiver\_rdt.Receiver attribute}}

\begin{fulllineitems}
\phantomsection\label{\detokenize{modules:receiver_rdt.Receiver.max_seq}}
\pysigstartsignatures
\pysigline
{\sphinxbfcode{\sphinxupquote{max\_seq}}}
\pysigstopsignatures
\sphinxAtStartPar
the highest sequence number known to the receiver

\end{fulllineitems}

\index{add\_packet() (receiver\_rdt.Receiver method)@\spxentry{add\_packet()}\spxextra{receiver\_rdt.Receiver method}}

\begin{fulllineitems}
\phantomsection\label{\detokenize{modules:receiver_rdt.Receiver.add_packet}}
\pysigstartsignatures
\pysiglinewithargsret
{\sphinxbfcode{\sphinxupquote{add\_packet}}}
{\sphinxparam{\DUrole{n}{file\_id}}\sphinxparamcomma \sphinxparam{\DUrole{n}{seq\_num}}\sphinxparamcomma \sphinxparam{\DUrole{n}{data\_str}}\sphinxparamcomma \sphinxparam{\DUrole{n}{expand\_pkts}}}
{}
\pysigstopsignatures
\sphinxAtStartPar
Given file\_id, seq\_num, data\_str, place the data into the correct spot
in ‘info{[}“packets”{]}’. If expand\_pkts is True, we enlarge ‘info{[}“packets”{]}’
up to seq\_num.
\begin{quote}\begin{description}
\sphinxlineitem{Parameters}\begin{itemize}
\item {} 
\sphinxAtStartPar
\sphinxstyleliteralstrong{\sphinxupquote{file\_id}} (\sphinxstyleliteralemphasis{\sphinxupquote{String}}) \textendash{} the identifier of the inbound file

\item {} 
\sphinxAtStartPar
\sphinxstyleliteralstrong{\sphinxupquote{seq\_num}} (\sphinxstyleliteralemphasis{\sphinxupquote{int}}) \textendash{} sequence number of this chunk

\item {} 
\sphinxAtStartPar
\sphinxstyleliteralstrong{\sphinxupquote{data\_str}} (\sphinxstyleliteralemphasis{\sphinxupquote{String}}) \textendash{} the chunk contents

\item {} 
\sphinxAtStartPar
\sphinxstyleliteralstrong{\sphinxupquote{expand\_pkts}} (\sphinxstyleliteralemphasis{\sphinxupquote{bool}}) \textendash{} True if seq\_num \textgreater{}= info{[}‘max\_seq’{]}, meaning we may need
to extend the packets array

\end{itemize}

\end{description}\end{quote}

\end{fulllineitems}

\index{base\_seq (receiver\_rdt.Receiver attribute)@\spxentry{base\_seq}\spxextra{receiver\_rdt.Receiver attribute}}

\begin{fulllineitems}
\phantomsection\label{\detokenize{modules:id0}}
\pysigstartsignatures
\pysigline
{\sphinxbfcode{\sphinxupquote{base\_seq}}\sphinxbfcode{\sphinxupquote{\DUrole{w}{ }\DUrole{p}{=}\DUrole{w}{ }\sphinxhyphen{}1}}}
\pysigstopsignatures
\end{fulllineitems}

\index{finalize\_file() (receiver\_rdt.Receiver method)@\spxentry{finalize\_file()}\spxextra{receiver\_rdt.Receiver method}}

\begin{fulllineitems}
\phantomsection\label{\detokenize{modules:receiver_rdt.Receiver.finalize_file}}
\pysigstartsignatures
\pysiglinewithargsret
{\sphinxbfcode{\sphinxupquote{finalize\_file}}}
{\sphinxparam{\DUrole{n}{file\_id}}}
{}
\pysigstopsignatures
\sphinxAtStartPar
Writes out the collected packets for ‘file\_id’ to \textless{}file\_id\textgreater{}\_torrent.txt,
then clears them from self.active\_files.
\begin{quote}\begin{description}
\sphinxlineitem{Parameters}
\sphinxAtStartPar
\sphinxstyleliteralstrong{\sphinxupquote{file\_id}} (\sphinxstyleliteralemphasis{\sphinxupquote{String}}) \textendash{} the unique identifier for the file being transferred

\end{description}\end{quote}

\end{fulllineitems}

\index{listen\_for\_requests() (receiver\_rdt.Receiver method)@\spxentry{listen\_for\_requests()}\spxextra{receiver\_rdt.Receiver method}}

\begin{fulllineitems}
\phantomsection\label{\detokenize{modules:receiver_rdt.Receiver.listen_for_requests}}
\pysigstartsignatures
\pysiglinewithargsret
{\sphinxbfcode{\sphinxupquote{listen\_for\_requests}}}
{\sphinxparam{\DUrole{n}{exch\_req\_queue}}}
{}
\pysigstopsignatures
\sphinxAtStartPar
Waits for request from other peers from self.soc, verifies data and verifies requests
Runs as long as the p2p\_command is running

\end{fulllineitems}

\index{max\_seq (receiver\_rdt.Receiver attribute)@\spxentry{max\_seq}\spxextra{receiver\_rdt.Receiver attribute}}

\begin{fulllineitems}
\phantomsection\label{\detokenize{modules:id1}}
\pysigstartsignatures
\pysigline
{\sphinxbfcode{\sphinxupquote{max\_seq}}\sphinxbfcode{\sphinxupquote{\DUrole{w}{ }\DUrole{p}{=}\DUrole{w}{ }\sphinxhyphen{}1}}}
\pysigstopsignatures
\end{fulllineitems}

\index{packets (receiver\_rdt.Receiver attribute)@\spxentry{packets}\spxextra{receiver\_rdt.Receiver attribute}}

\begin{fulllineitems}
\phantomsection\label{\detokenize{modules:id2}}
\pysigstartsignatures
\pysigline
{\sphinxbfcode{\sphinxupquote{packets}}\sphinxbfcode{\sphinxupquote{\DUrole{w}{ }\DUrole{p}{=}\DUrole{w}{ }{[}{]}}}}
\pysigstopsignatures
\end{fulllineitems}

\index{rebase\_packets() (receiver\_rdt.Receiver method)@\spxentry{rebase\_packets()}\spxextra{receiver\_rdt.Receiver method}}

\begin{fulllineitems}
\phantomsection\label{\detokenize{modules:receiver_rdt.Receiver.rebase_packets}}
\pysigstartsignatures
\pysiglinewithargsret
{\sphinxbfcode{\sphinxupquote{rebase\_packets}}}
{\sphinxparam{\DUrole{n}{file\_id}}\sphinxparamcomma \sphinxparam{\DUrole{n}{seq\_num}}\sphinxparamcomma \sphinxparam{\DUrole{n}{data\_str}}}
{}
\pysigstopsignatures
\sphinxAtStartPar
Given file\_id and a chunk’s sequence number is smaller than base\_seq,
rebase so that ‘seq\_num’ becomes the new base\_seq and put ‘data\_str’
at index 0.
\begin{quote}\begin{description}
\sphinxlineitem{Parameters}\begin{itemize}
\item {} 
\sphinxAtStartPar
\sphinxstyleliteralstrong{\sphinxupquote{file\_id}} (\sphinxstyleliteralemphasis{\sphinxupquote{String}}) \textendash{} the identifier of the inbound file

\item {} 
\sphinxAtStartPar
\sphinxstyleliteralstrong{\sphinxupquote{seq\_num}} (\sphinxstyleliteralemphasis{\sphinxupquote{int}}) \textendash{} the inbound packet’s sequence number

\item {} 
\sphinxAtStartPar
\sphinxstyleliteralstrong{\sphinxupquote{data\_str}} (\sphinxstyleliteralemphasis{\sphinxupquote{String}}) \textendash{} the actual file data chunk

\end{itemize}

\end{description}\end{quote}

\end{fulllineitems}

\index{set\_timeout() (receiver\_rdt.Receiver method)@\spxentry{set\_timeout()}\spxextra{receiver\_rdt.Receiver method}}

\begin{fulllineitems}
\phantomsection\label{\detokenize{modules:receiver_rdt.Receiver.set_timeout}}
\pysigstartsignatures
\pysiglinewithargsret
{\sphinxbfcode{\sphinxupquote{set\_timeout}}}
{}
{}
\pysigstopsignatures
\sphinxAtStartPar
Optional method to signal that this receiver should time out.

\end{fulllineitems}

\index{timeout (receiver\_rdt.Receiver attribute)@\spxentry{timeout}\spxextra{receiver\_rdt.Receiver attribute}}

\begin{fulllineitems}
\phantomsection\label{\detokenize{modules:receiver_rdt.Receiver.timeout}}
\pysigstartsignatures
\pysigline
{\sphinxbfcode{\sphinxupquote{timeout}}\sphinxbfcode{\sphinxupquote{\DUrole{w}{ }\DUrole{p}{=}\DUrole{w}{ }None}}}
\pysigstopsignatures
\end{fulllineitems}


\end{fulllineitems}

\index{convert\_sender\_payload() (in module receiver\_rdt)@\spxentry{convert\_sender\_payload()}\spxextra{in module receiver\_rdt}}

\begin{fulllineitems}
\phantomsection\label{\detokenize{modules:receiver_rdt.convert_sender_payload}}
\pysigstartsignatures
\pysiglinewithargsret
{\sphinxcode{\sphinxupquote{receiver\_rdt.}}\sphinxbfcode{\sphinxupquote{convert\_sender\_payload}}}
{\sphinxparam{\DUrole{n}{data}}}
{}
\pysigstopsignatures
\sphinxAtStartPar
Decodes packet payload to retrieve sequence number and message of packet
\begin{quote}\begin{description}
\sphinxlineitem{Parameters}
\sphinxAtStartPar
\sphinxstyleliteralstrong{\sphinxupquote{data}} (\sphinxstyleliteralemphasis{\sphinxupquote{Bytes}}) \textendash{} sequence of Bytes to decode

\sphinxlineitem{Returns}
\sphinxAtStartPar
send\_seq, sequence number of packet

\sphinxlineitem{Return type}
\sphinxAtStartPar
Bytes

\sphinxlineitem{Returns}
\sphinxAtStartPar
msg, data from packet

\sphinxlineitem{Return type}
\sphinxAtStartPar
String

\end{description}\end{quote}

\end{fulllineitems}

\index{make\_checksum() (in module receiver\_rdt)@\spxentry{make\_checksum()}\spxextra{in module receiver\_rdt}}

\begin{fulllineitems}
\phantomsection\label{\detokenize{modules:receiver_rdt.make_checksum}}
\pysigstartsignatures
\pysiglinewithargsret
{\sphinxcode{\sphinxupquote{receiver\_rdt.}}\sphinxbfcode{\sphinxupquote{make\_checksum}}}
{\sphinxparam{\DUrole{n}{data}}}
{}
\pysigstopsignatures
\sphinxAtStartPar
Forms checksum from data using crc32 function from zlib library
\begin{quote}\begin{description}
\sphinxlineitem{Parameters}
\sphinxAtStartPar
\sphinxstyleliteralstrong{\sphinxupquote{data}} (\sphinxstyleliteralemphasis{\sphinxupquote{Bytes}}) \textendash{} sequence of Bytes to calculate checksum

\sphinxlineitem{Returns}
\sphinxAtStartPar
checksum of data

\sphinxlineitem{Return type}
\sphinxAtStartPar
Bytes

\end{description}\end{quote}

\end{fulllineitems}

\index{make\_packet() (in module receiver\_rdt)@\spxentry{make\_packet()}\spxextra{in module receiver\_rdt}}

\begin{fulllineitems}
\phantomsection\label{\detokenize{modules:receiver_rdt.make_packet}}
\pysigstartsignatures
\pysiglinewithargsret
{\sphinxcode{\sphinxupquote{receiver\_rdt.}}\sphinxbfcode{\sphinxupquote{make\_packet}}}
{\sphinxparam{\DUrole{n}{seq\_num}}\sphinxparamcomma \sphinxparam{\DUrole{n}{msg}}}
{}
\pysigstopsignatures
\sphinxAtStartPar
Forms packet by combining calculated checksum and formed payload
\begin{quote}\begin{description}
\sphinxlineitem{Parameters}\begin{itemize}
\item {} 
\sphinxAtStartPar
\sphinxstyleliteralstrong{\sphinxupquote{seq\_num}} (\sphinxstyleliteralemphasis{\sphinxupquote{int}}) \textendash{} int to convert to bytes

\item {} 
\sphinxAtStartPar
\sphinxstyleliteralstrong{\sphinxupquote{msg}} (\sphinxstyleliteralemphasis{\sphinxupquote{String}}) \textendash{} characters to encode

\end{itemize}

\sphinxlineitem{Returns}
\sphinxAtStartPar
payload, sequence of bytes containing seq\_num and msg

\sphinxlineitem{Return type}
\sphinxAtStartPar
Bytes

\end{description}\end{quote}

\end{fulllineitems}

\index{make\_receiver\_payload() (in module receiver\_rdt)@\spxentry{make\_receiver\_payload()}\spxextra{in module receiver\_rdt}}

\begin{fulllineitems}
\phantomsection\label{\detokenize{modules:receiver_rdt.make_receiver_payload}}
\pysigstartsignatures
\pysiglinewithargsret
{\sphinxcode{\sphinxupquote{receiver\_rdt.}}\sphinxbfcode{\sphinxupquote{make\_receiver\_payload}}}
{\sphinxparam{\DUrole{n}{seq\_num}}\sphinxparamcomma \sphinxparam{\DUrole{n}{msg}}}
{}
\pysigstopsignatures
\sphinxAtStartPar
Forms packet payload by encoding sequence number and message of packet
\begin{quote}\begin{description}
\sphinxlineitem{Parameters}\begin{itemize}
\item {} 
\sphinxAtStartPar
\sphinxstyleliteralstrong{\sphinxupquote{seq\_num}} (\sphinxstyleliteralemphasis{\sphinxupquote{int}}) \textendash{} int to convert to bytes

\item {} 
\sphinxAtStartPar
\sphinxstyleliteralstrong{\sphinxupquote{msg}} (\sphinxstyleliteralemphasis{\sphinxupquote{String}}) \textendash{} characters to encode

\end{itemize}

\sphinxlineitem{Returns}
\sphinxAtStartPar
payload, sequence of bytes containing seq\_num and msg

\sphinxlineitem{Return type}
\sphinxAtStartPar
Bytes

\end{description}\end{quote}

\end{fulllineitems}

\index{verify\_integrity() (in module receiver\_rdt)@\spxentry{verify\_integrity()}\spxextra{in module receiver\_rdt}}

\begin{fulllineitems}
\phantomsection\label{\detokenize{modules:receiver_rdt.verify_integrity}}
\pysigstartsignatures
\pysiglinewithargsret
{\sphinxcode{\sphinxupquote{receiver\_rdt.}}\sphinxbfcode{\sphinxupquote{verify\_integrity}}}
{\sphinxparam{\DUrole{n}{sent\_chksum}}\sphinxparamcomma \sphinxparam{\DUrole{n}{data}}}
{}
\pysigstopsignatures
\sphinxAtStartPar
Verifies checksum from received packet
\begin{quote}\begin{description}
\sphinxlineitem{Parameters}\begin{itemize}
\item {} 
\sphinxAtStartPar
\sphinxstyleliteralstrong{\sphinxupquote{sent\_chksum}} (\sphinxstyleliteralemphasis{\sphinxupquote{Bytes}}) \textendash{} received checksum with length of 8 bytes

\item {} 
\sphinxAtStartPar
\sphinxstyleliteralstrong{\sphinxupquote{data}} (\sphinxstyleliteralemphasis{\sphinxupquote{Bytes}}) \textendash{} sequence of bytes to calculate checksum with

\end{itemize}

\sphinxlineitem{Returns}
\sphinxAtStartPar
if sent\_chksum is the exact same as calculated checksum

\sphinxlineitem{Return type}
\sphinxAtStartPar
Boolean

\end{description}\end{quote}

\end{fulllineitems}



\subsection{sender\_rdt}
\label{\detokenize{modules:module-sender_rdt}}\label{\detokenize{modules:sender-rdt}}\index{module@\spxentry{module}!sender\_rdt@\spxentry{sender\_rdt}}\index{sender\_rdt@\spxentry{sender\_rdt}!module@\spxentry{module}}\index{Sender (class in sender\_rdt)@\spxentry{Sender}\spxextra{class in sender\_rdt}}

\begin{fulllineitems}
\phantomsection\label{\detokenize{modules:sender_rdt.Sender}}
\pysigstartsignatures
\pysiglinewithargsret
{\sphinxbfcode{\sphinxupquote{class\DUrole{w}{ }}}\sphinxcode{\sphinxupquote{sender\_rdt.}}\sphinxbfcode{\sphinxupquote{Sender}}}
{\sphinxparam{\DUrole{n}{soc}}\sphinxparamcomma \sphinxparam{\DUrole{n}{ip}}\sphinxparamcomma \sphinxparam{\DUrole{n}{port}}\sphinxparamcomma \sphinxparam{\DUrole{n}{file\_id}}}
{}
\pysigstopsignatures
\sphinxAtStartPar
Bases: \sphinxcode{\sphinxupquote{object}}

\sphinxAtStartPar
Sender, a class with defined behavior to send data to a receiver
\index{packets (sender\_rdt.Sender attribute)@\spxentry{packets}\spxextra{sender\_rdt.Sender attribute}}

\begin{fulllineitems}
\phantomsection\label{\detokenize{modules:sender_rdt.Sender.packets}}
\pysigstartsignatures
\pysigline
{\sphinxbfcode{\sphinxupquote{packets}}}
\pysigstopsignatures
\sphinxAtStartPar
Array of 3 object arrays containing:

\end{fulllineitems}



\begin{fulllineitems}

\pysigstartsignatures
\pysigline
{\sphinxbfcode{\sphinxupquote{{[}formed~byte~packet,~boolean~ack,~Timeout~retransmission~thread{]}}}}
\pysigstopsignatures
\end{fulllineitems}

\index{soc (sender\_rdt.Sender attribute)@\spxentry{soc}\spxextra{sender\_rdt.Sender attribute}}

\begin{fulllineitems}
\phantomsection\label{\detokenize{modules:sender_rdt.Sender.soc}}
\pysigstartsignatures
\pysigline
{\sphinxbfcode{\sphinxupquote{soc}}}
\pysigstopsignatures
\sphinxAtStartPar
socket that sender uses to send data over

\end{fulllineitems}

\index{ip (sender\_rdt.Sender attribute)@\spxentry{ip}\spxextra{sender\_rdt.Sender attribute}}

\begin{fulllineitems}
\phantomsection\label{\detokenize{modules:sender_rdt.Sender.ip}}
\pysigstartsignatures
\pysigline
{\sphinxbfcode{\sphinxupquote{ip}}}
\pysigstopsignatures
\sphinxAtStartPar
ip address to send data to

\end{fulllineitems}

\index{port (sender\_rdt.Sender attribute)@\spxentry{port}\spxextra{sender\_rdt.Sender attribute}}

\begin{fulllineitems}
\phantomsection\label{\detokenize{modules:sender_rdt.Sender.port}}
\pysigstartsignatures
\pysigline
{\sphinxbfcode{\sphinxupquote{port}}}
\pysigstopsignatures
\sphinxAtStartPar
port number to send data to

\end{fulllineitems}

\index{base\_seq (sender\_rdt.Sender attribute)@\spxentry{base\_seq}\spxextra{sender\_rdt.Sender attribute}}

\begin{fulllineitems}
\phantomsection\label{\detokenize{modules:sender_rdt.Sender.base_seq}}
\pysigstartsignatures
\pysigline
{\sphinxbfcode{\sphinxupquote{base\_seq}}}
\pysigstopsignatures
\sphinxAtStartPar
the lowest sequence number to index by

\end{fulllineitems}

\index{arrange\_pkts() (sender\_rdt.Sender method)@\spxentry{arrange\_pkts()}\spxextra{sender\_rdt.Sender method}}

\begin{fulllineitems}
\phantomsection\label{\detokenize{modules:sender_rdt.Sender.arrange_pkts}}
\pysigstartsignatures
\pysiglinewithargsret
{\sphinxbfcode{\sphinxupquote{arrange\_pkts}}}
{\sphinxparam{\DUrole{n}{data}}}
{}
\pysigstopsignatures
\sphinxAtStartPar
Given chunks of data, populate each entry of Sender packets with
packet, False (for acknowledgement), thread.Timer for timeout and retransmit
\begin{quote}\begin{description}
\sphinxlineitem{Parameters}
\sphinxAtStartPar
\sphinxstyleliteralstrong{\sphinxupquote{data}} (\sphinxstyleliteralemphasis{\sphinxupquote{Array}}\sphinxstyleliteralemphasis{\sphinxupquote{ of }}\sphinxstyleliteralemphasis{\sphinxupquote{Strings}}) \textendash{} array of chunks of data

\end{description}\end{quote}

\end{fulllineitems}

\index{find\_recv\_base\_window() (sender\_rdt.Sender method)@\spxentry{find\_recv\_base\_window()}\spxextra{sender\_rdt.Sender method}}

\begin{fulllineitems}
\phantomsection\label{\detokenize{modules:sender_rdt.Sender.find_recv_base_window}}
\pysigstartsignatures
\pysiglinewithargsret
{\sphinxbfcode{\sphinxupquote{find\_recv\_base\_window}}}
{\sphinxparam{\DUrole{n}{window\_size}}}
{}
\pysigstopsignatures
\sphinxAtStartPar
Given window size and Sender packets,
find the closest unacknowledged packet and calculate the window
\begin{quote}\begin{description}
\sphinxlineitem{Parameters}
\sphinxAtStartPar
\sphinxstyleliteralstrong{\sphinxupquote{window\_size}} (\sphinxstyleliteralemphasis{\sphinxupquote{int}}) \textendash{} size of window

\end{description}\end{quote}

\end{fulllineitems}

\index{make\_packets() (sender\_rdt.Sender method)@\spxentry{make\_packets()}\spxextra{sender\_rdt.Sender method}}

\begin{fulllineitems}
\phantomsection\label{\detokenize{modules:sender_rdt.Sender.make_packets}}
\pysigstartsignatures
\pysiglinewithargsret
{\sphinxbfcode{\sphinxupquote{make\_packets}}}
{\sphinxparam{\DUrole{n}{exch\_path}}\sphinxparamcomma \sphinxparam{\DUrole{n}{chunk\_size}}}
{}
\pysigstopsignatures
\sphinxAtStartPar
Forms packets from file by splitting file into chunks
\begin{quote}\begin{description}
\sphinxlineitem{Parameters}\begin{itemize}
\item {} 
\sphinxAtStartPar
\sphinxstyleliteralstrong{\sphinxupquote{file\_name}} (\sphinxstyleliteralemphasis{\sphinxupquote{String}}) \textendash{} String containing name of file to send

\item {} 
\sphinxAtStartPar
\sphinxstyleliteralstrong{\sphinxupquote{chunk\_size}} (\sphinxstyleliteralemphasis{\sphinxupquote{int}}) \textendash{} number of characters to fit in a chunk from file

\end{itemize}

\sphinxlineitem{Returns}
\sphinxAtStartPar
pkts, array of character chunks from file

\sphinxlineitem{Return type}
\sphinxAtStartPar
Array

\end{description}\end{quote}

\end{fulllineitems}

\index{packets (sender\_rdt.Sender attribute)@\spxentry{packets}\spxextra{sender\_rdt.Sender attribute}}

\begin{fulllineitems}
\phantomsection\label{\detokenize{modules:id3}}
\pysigstartsignatures
\pysigline
{\sphinxbfcode{\sphinxupquote{packets}}\sphinxbfcode{\sphinxupquote{\DUrole{w}{ }\DUrole{p}{=}\DUrole{w}{ }None}}}
\pysigstopsignatures
\end{fulllineitems}

\index{run\_sender() (sender\_rdt.Sender method)@\spxentry{run\_sender()}\spxextra{sender\_rdt.Sender method}}

\begin{fulllineitems}
\phantomsection\label{\detokenize{modules:sender_rdt.Sender.run_sender}}
\pysigstartsignatures
\pysiglinewithargsret
{\sphinxbfcode{\sphinxupquote{run\_sender}}}
{}
{}
\pysigstopsignatures
\sphinxAtStartPar
Sends packets using Selective Repeat. Only creates timers once per packet.

\end{fulllineitems}

\index{send\_pkt() (sender\_rdt.Sender method)@\spxentry{send\_pkt()}\spxextra{sender\_rdt.Sender method}}

\begin{fulllineitems}
\phantomsection\label{\detokenize{modules:sender_rdt.Sender.send_pkt}}
\pysigstartsignatures
\pysiglinewithargsret
{\sphinxbfcode{\sphinxupquote{send\_pkt}}}
{\sphinxparam{\DUrole{n}{seq\_num}}}
{}
\pysigstopsignatures
\sphinxAtStartPar
Retransmits packet after timeout by thread.Timer and resets timeout
\begin{quote}\begin{description}
\sphinxlineitem{Parameters}
\sphinxAtStartPar
\sphinxstyleliteralstrong{\sphinxupquote{seq\_num}} (\sphinxstyleliteralemphasis{\sphinxupquote{int}}) \textendash{} sequence number to retransmit

\end{description}\end{quote}

\end{fulllineitems}

\index{setup\_exchange() (sender\_rdt.Sender method)@\spxentry{setup\_exchange()}\spxextra{sender\_rdt.Sender method}}

\begin{fulllineitems}
\phantomsection\label{\detokenize{modules:sender_rdt.Sender.setup_exchange}}
\pysigstartsignatures
\pysiglinewithargsret
{\sphinxbfcode{\sphinxupquote{setup\_exchange}}}
{\sphinxparam{\DUrole{n}{exch\_path}}}
{}
\pysigstopsignatures
\end{fulllineitems}


\end{fulllineitems}

\index{convert\_ack\_payload() (in module sender\_rdt)@\spxentry{convert\_ack\_payload()}\spxextra{in module sender\_rdt}}

\begin{fulllineitems}
\phantomsection\label{\detokenize{modules:sender_rdt.convert_ack_payload}}
\pysigstartsignatures
\pysiglinewithargsret
{\sphinxcode{\sphinxupquote{sender\_rdt.}}\sphinxbfcode{\sphinxupquote{convert\_ack\_payload}}}
{\sphinxparam{\DUrole{n}{data}}}
{}
\pysigstopsignatures
\sphinxAtStartPar
Parses a receiver ACK payload (just a sequence number + “ACK”)
\begin{quote}\begin{description}
\sphinxlineitem{Parameters}
\sphinxAtStartPar
\sphinxstyleliteralstrong{\sphinxupquote{data}} \textendash{} sequence of bytes

\sphinxlineitem{Returns}
\sphinxAtStartPar
seq\_num, message

\end{description}\end{quote}

\end{fulllineitems}

\index{convert\_receiver\_payload() (in module sender\_rdt)@\spxentry{convert\_receiver\_payload()}\spxextra{in module sender\_rdt}}

\begin{fulllineitems}
\phantomsection\label{\detokenize{modules:sender_rdt.convert_receiver_payload}}
\pysigstartsignatures
\pysiglinewithargsret
{\sphinxcode{\sphinxupquote{sender\_rdt.}}\sphinxbfcode{\sphinxupquote{convert\_receiver\_payload}}}
{\sphinxparam{\DUrole{n}{data}}}
{}
\pysigstopsignatures
\sphinxAtStartPar
Decodes packet payload to retrieve sequence number and message of packet
\begin{quote}\begin{description}
\sphinxlineitem{Parameters}
\sphinxAtStartPar
\sphinxstyleliteralstrong{\sphinxupquote{data}} (\sphinxstyleliteralemphasis{\sphinxupquote{Bytes}}) \textendash{} sequence of Bytes to decode

\sphinxlineitem{Returns}
\sphinxAtStartPar
send\_seq, sequence number of packet

\sphinxlineitem{Return type}
\sphinxAtStartPar
Bytes

\sphinxlineitem{Returns}
\sphinxAtStartPar
msg, data from packet

\sphinxlineitem{Return type}
\sphinxAtStartPar
String

\end{description}\end{quote}

\end{fulllineitems}

\index{make\_checksum() (in module sender\_rdt)@\spxentry{make\_checksum()}\spxextra{in module sender\_rdt}}

\begin{fulllineitems}
\phantomsection\label{\detokenize{modules:sender_rdt.make_checksum}}
\pysigstartsignatures
\pysiglinewithargsret
{\sphinxcode{\sphinxupquote{sender\_rdt.}}\sphinxbfcode{\sphinxupquote{make\_checksum}}}
{\sphinxparam{\DUrole{n}{data}}}
{}
\pysigstopsignatures
\sphinxAtStartPar
Forms checksum from data using crc32 function from zlib library
\begin{quote}\begin{description}
\sphinxlineitem{Parameters}
\sphinxAtStartPar
\sphinxstyleliteralstrong{\sphinxupquote{data}} (\sphinxstyleliteralemphasis{\sphinxupquote{Bytes}}) \textendash{} sequence of Bytes to calculate checksum

\sphinxlineitem{Returns}
\sphinxAtStartPar
checksum of data

\sphinxlineitem{Return type}
\sphinxAtStartPar
Bytes

\end{description}\end{quote}

\end{fulllineitems}

\index{make\_packet() (in module sender\_rdt)@\spxentry{make\_packet()}\spxextra{in module sender\_rdt}}

\begin{fulllineitems}
\phantomsection\label{\detokenize{modules:sender_rdt.make_packet}}
\pysigstartsignatures
\pysiglinewithargsret
{\sphinxcode{\sphinxupquote{sender\_rdt.}}\sphinxbfcode{\sphinxupquote{make\_packet}}}
{\sphinxparam{\DUrole{n}{seq\_num}}\sphinxparamcomma \sphinxparam{\DUrole{n}{msg}}\sphinxparamcomma \sphinxparam{\DUrole{n}{file\_id}}}
{}
\pysigstopsignatures
\sphinxAtStartPar
Forms packet by combining calculated checksum and formed payload
\begin{quote}\begin{description}
\sphinxlineitem{Parameters}\begin{itemize}
\item {} 
\sphinxAtStartPar
\sphinxstyleliteralstrong{\sphinxupquote{seq\_num}} (\sphinxstyleliteralemphasis{\sphinxupquote{int}}) \textendash{} int to convert to bytes

\item {} 
\sphinxAtStartPar
\sphinxstyleliteralstrong{\sphinxupquote{msg}} (\sphinxstyleliteralemphasis{\sphinxupquote{String}}) \textendash{} characters to encode

\end{itemize}

\sphinxlineitem{Returns}
\sphinxAtStartPar
payload, sequence of bytes containing seq\_num and msg

\sphinxlineitem{Return type}
\sphinxAtStartPar
Bytes

\end{description}\end{quote}

\end{fulllineitems}

\index{make\_sender\_payload() (in module sender\_rdt)@\spxentry{make\_sender\_payload()}\spxextra{in module sender\_rdt}}

\begin{fulllineitems}
\phantomsection\label{\detokenize{modules:sender_rdt.make_sender_payload}}
\pysigstartsignatures
\pysiglinewithargsret
{\sphinxcode{\sphinxupquote{sender\_rdt.}}\sphinxbfcode{\sphinxupquote{make\_sender\_payload}}}
{\sphinxparam{\DUrole{n}{seq\_num}}\sphinxparamcomma \sphinxparam{\DUrole{n}{msg}}\sphinxparamcomma \sphinxparam{\DUrole{n}{file\_id}}}
{}
\pysigstopsignatures
\sphinxAtStartPar
Forms packet payload by encoding sequence number and message of packet
\begin{quote}\begin{description}
\sphinxlineitem{Parameters}\begin{itemize}
\item {} 
\sphinxAtStartPar
\sphinxstyleliteralstrong{\sphinxupquote{seq\_num}} (\sphinxstyleliteralemphasis{\sphinxupquote{int}}) \textendash{} int to convert to bytes

\item {} 
\sphinxAtStartPar
\sphinxstyleliteralstrong{\sphinxupquote{msg}} (\sphinxstyleliteralemphasis{\sphinxupquote{String}}) \textendash{} characters to encode

\end{itemize}

\sphinxlineitem{Returns}
\sphinxAtStartPar
payload, sequence of bytes containing seq\_num and msg

\sphinxlineitem{Return type}
\sphinxAtStartPar
Bytes

\end{description}\end{quote}

\end{fulllineitems}

\index{verify\_integrity() (in module sender\_rdt)@\spxentry{verify\_integrity()}\spxextra{in module sender\_rdt}}

\begin{fulllineitems}
\phantomsection\label{\detokenize{modules:sender_rdt.verify_integrity}}
\pysigstartsignatures
\pysiglinewithargsret
{\sphinxcode{\sphinxupquote{sender\_rdt.}}\sphinxbfcode{\sphinxupquote{verify\_integrity}}}
{\sphinxparam{\DUrole{n}{sent\_chksum}}\sphinxparamcomma \sphinxparam{\DUrole{n}{data}}}
{}
\pysigstopsignatures
\sphinxAtStartPar
Verifies checksum from received packet
\begin{quote}\begin{description}
\sphinxlineitem{Parameters}\begin{itemize}
\item {} 
\sphinxAtStartPar
\sphinxstyleliteralstrong{\sphinxupquote{sent\_chksum}} (\sphinxstyleliteralemphasis{\sphinxupquote{Bytes}}) \textendash{} received checksum with length of 8 bytes

\item {} 
\sphinxAtStartPar
\sphinxstyleliteralstrong{\sphinxupquote{data}} (\sphinxstyleliteralemphasis{\sphinxupquote{Bytes}}) \textendash{} sequence of bytes to calculate checksum with

\end{itemize}

\sphinxlineitem{Returns}
\sphinxAtStartPar
if sent\_chksum is the exact same as calculated checksum

\sphinxlineitem{Return type}
\sphinxAtStartPar
Boolean

\end{description}\end{quote}

\end{fulllineitems}




\renewcommand{\indexname}{Index}
\printindex
\end{document}